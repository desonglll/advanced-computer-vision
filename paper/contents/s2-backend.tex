\section{前后端设计方案}

\subsection{整体架构}

本方案通过整合Django框架与姿势识别程序,实现前后端的协同工作,确保系统能够高效地处理用户请求并提供实时反馈。系统架构由以下几个核心模块组成:

\begin{enumerate}
    \item \textbf{前端模块:}
    前端部分采用HTML、CSS和JavaScript开发,利用前端框架如React或Vue.js提供用户友好的交互界面。用户通过网页界面上传图片,查看姿势识别的结果,且能够实时查看识别过程中的反馈信息。此外,前端还负责与后端进行数据交互,发送用户请求和展示姿势识别结果。

    \item \textbf{后端模块:}
    后端部分基于Django框架,负责处理来自前端的请求,执行业务逻辑,调用姿势识别程序,并返回识别结果。后端模块的主要职责包括:接收用户上传的图片文件、调用YOLO姿势识别模型进行处理、将识别结果返回给前端展示。后端还负责实现用户认证、权限管理及其他核心功能。

    \item \textbf{模型处理模块:}
    模型处理模块包括YOLO姿势识别模型及其相关算法。此模块负责接收上传的图片,使用预训练的YOLO模型进行姿势检测与识别,提取关键点坐标、姿势角度等信息,并将结果传递给后端模块进行进一步处理。该模块可能与OpenCV等图像处理库协同工作,确保高效的图像预处理与实时识别。

    \item \textbf{数据库模块:}
    数据库模块采用PostgreSQL作为数据存储解决方案。该模块负责存储用户数据、上传的图片信息以及姿势识别结果。数据库设计包含用户表、图片表、识别结果表等,确保数据的高效存储与快速查询。此外,数据库模块还支持用户信息的增删改查操作,用于管理用户资料与相关数据。
\end{enumerate}

系统的整体流程为:用户通过前端模块上传图片,后端接收图片并调用模型处理模块进行姿势识别,最后将识别结果与相关数据存入数据库,并返回给前端展示。这种架构实现了高效的数据流转和协同工作,支持系统的可扩展性与可维护性。


\subsection{前端设计}

\subsubsection{主要架构}

系统的前端界面分为几个主要区域:顶部菜单栏提供基础的导航功能,如上传图片、历史记录查询和帮助选项;中央核心区域展示OpenCV实时画面,主要显示姿势识别过程中的骨架关键点或其他视觉反馈;右侧功能区域包含摄像头控制和参数设置,允许用户启动或关闭摄像头、调整摄像头设置以及配置模型参数;底部结果区域则提供实时姿势状态、系统运行日志和间隔检测分析结果等信息。

技术架构方面,前端使用HTML、CSS、JavaScript与前端框架(如React或Vue.js)结合实现动态界面更新。后端采用Django框架,处理业务逻辑和数据交互。姿势识别功能通过集成YOLO模型或其他深度学习模型,并结合OpenCV进行实现。

\subsubsection{主要功能}

系统的主要功能包括图片上传、实时反馈、参数配置、结果展示、历史查询和无刷新交互。用户可以通过菜单栏快速上传图片,支持多种格式。上传后,图片会传递到后端并启动模型进行分析。实时反馈方面,OpenCV实时画面动态显示检测过程,用户可以看到骨架图或其他标注信息,同时摄像头实时捕获图像流,帮助用户了解当前状态。

参数配置部分,用户可以调节摄像头相关参数,如分辨率和曝光度;姿势识别参数,如模型选择和置信度设置,也可以进行调整。结果展示方面,检测完成后,前端会实时展示关键点位置信息和预测分类结果。历史查询功能允许用户通过筛选条件查询识别历史,数据通过AJAX动态加载优化查询体验。无刷新交互利用AJAX技术,使得用户无需刷新页面即可更新或获取新的检测结果。

\subsubsection{系统优势}

系统具有模块化设计、实时性和可扩展性等优势。各功能区域相对独立,有利于系统的维护和扩展。通过结合OpenCV和YOLO模型,系统能够实现高效的姿势检测,满足不同应用场景的需求。界面和功能可以根据需求进行调整,适应健康监测、运动指导等应用场景。

\subsubsection{用户界面}

前端页面提供以下主要功能,确保用户能够方便地与系统交互并获取姿势识别的结果:

\begin{enumerate}
    \item \textbf{图片上传:}
    用户可以通过简洁的表单界面上传图片文件。该功能通过HTML表单和JavaScript处理,支持多种图片格式(如JPG、PNG)。用户选择文件后,点击上传按钮,系统会自动将图片发送到后端进行处理。为了提升用户体验,前端还会提供上传进度条,显示图片上传的状态,确保用户可以及时了解上传的进度。

    \item \textbf{识别结果展示:}
    在图片上传并进行姿势识别后,系统会展示包括关键点数据在内的识别结果。识别结果将通过动态页面更新的方式呈现,用户可以在页面上看到姿势识别后的骨架图,并且每个关键点的坐标和类别会以表格或图形的方式展示。此外,前端会根据模型预测的分类结果显示相应的分类标签(如"正常姿势"、"错误姿势"等)。这些结果将实时显示,帮助用户了解上传图片的分析信息。

    \item \textbf{数据查询:}
    系统提供查询功能,允许用户查看历史上传记录及相关识别结果。用户可以通过日期、文件名、分类标签等条件进行筛选。每条历史记录会展示上传的图片、识别结果及对应的时间戳。查询功能通过后端与数据库的交互实现,确保用户能够快速查找到所需数据。查询界面将提供分页显示和搜索功能,优化查询效率,特别是在大量数据的情况下,用户仍能流畅操作。
\end{enumerate}

以上功能通过精心设计的用户界面进行呈现,结合前端技术(如HTML5、CSS3、JavaScript、React等)实现,保证了系统的交互性、易用性和高效性。用户能够快速上传图片、查看姿势识别结果,并查询历史记录,提升了整体体验。


\subsubsection{交互流程}

系统通过AJAX请求与Django后端进行交互,确保前端页面在进行数据交互时无需刷新,从而显著提升了用户体验。整个交互流程如下:

\begin{enumerate}
    \item \textbf{前端界面:}

    \begin{figure}[H]
        \centering
        \includegraphics[width=0.8\textwidth]{imgs/frontend-1.png}
        \caption{前端界面概览}
        \label{fig:file-tree}
    \end{figure}
    % 用户选择并上传图片后,前端通过AJAX向Django后端发送POST请求。请求体中包含图片文件和相关的元数据(如图片名、文件类型等)。AJAX请求的发送不会导致页面刷新,用户可以继续浏览页面,而后台则开始处理图片。前端通过AJAX的回调函数监控上传过程,实时更新进度条,直观地向用户显示上传进度。上传完成后,后端将图片路径存入数据库,并将图片传送给姿势识别模型进行分析。

    \item \textbf{姿势识别:}
    上传完成后,Django后端使用已加载的姿势识别模型(如YOLO或其他深度学习模型)进行处理。识别结果,包括33个关键点的位置及预测分类,会通过AJAX请求返回给前端。后端将处理结果封装为JSON格式,并通过AJAX响应发送给前端,数据包括图片文件路径、识别到的关键点坐标、分类结果等信息。前端接收到数据后,通过JavaScript处理并动态更新页面,显示识别结果,且无需刷新页面。

    \item \textbf{结果展示与查询:}
    用户上传的图片及其识别结果存储在数据库中。用户可以通过前端的查询界面发起AJAX请求,根据条件(如上传时间、分类标签、图片名等)检索历史记录。每次查询时,AJAX请求会向Django后端发送相应的查询参数,后端会从数据库中获取符合条件的记录,并通过JSON格式将结果返回。前端接收到查询结果后,动态更新页面内容,展示符合条件的图片及识别结果,用户无需手动刷新页面。

    \item \textbf{无刷新更新:}
    AJAX请求的核心在于其无刷新特性。前端页面无需全页刷新即可获取和展示数据。这通过在页面中局部更新内容来实现,例如更新识别结果的图像和表格数据。这种交互方式提升了用户体验,减少了页面加载时间和服务器压力,使得页面更加流畅和高效。

\end{enumerate}

通过AJAX与Django后端的交互,系统实现了实时响应和无刷新更新,大大增强了交互的流畅度和用户的使用体验。该流程不仅提高了系统的响应速度,还能处理复杂的多步操作,用户无需等待整个页面重新加载即可获取实时结果,进一步优化了前后端协作效率。


\subsection{后端设计}

\subsubsection{Django应用模块}

后端由以下主要模块构成:

\begin{enumerate}
    \item \textbf{用户管理模块:}
    该模块负责处理用户的注册、登录及权限管理。用户通过前端页面提交注册信息后,后端通过Django的认证系统进行处理,确保用户信息的安全性和完整性。用户登录时,后端验证用户的身份,使用会话(Session)管理用户的登录状态。该模块还包括权限控制,确保不同角色的用户能够访问特定的资源。例如,管理员可访问所有图片及其识别结果,而普通用户只能访问自己的数据。

    \item \textbf{图片上传模块:}
    用户通过前端上传图片时,该模块负责接收图片文件并将其保存到服务器的指定路径。图片通过AJAX请求提交给后端,后端使用Django的文件处理功能(如`FileField`)接收上传的文件,并将其存储到服务器的本地或云存储中。同时,后端会对上传的图片进行验证,确保文件类型和大小符合要求。成功上传后,图片路径和相关信息会保存到数据库中,以便后续检索和展示。

    \item \textbf{姿势识别模块:}
    该模块负责调用姿势识别程序(如YOLO或其他深度学习模型)处理用户上传的图片并返回识别结果。处理过程中,后端会将上传的图片传递给姿势识别模型进行分析,提取出图片中的人体关键点坐标以及预测的姿势分类。识别结果以JSON格式返回,包括33个关键点的坐标(x, y, z)和可见性(visibility)。返回的识别结果会被传递至前端,供用户查看。

    \item \textbf{数据管理模块:}
    该模块负责将识别结果保存到数据库中,并提供查询接口。所有上传的图片及其识别结果都会被存储在PostgreSQL数据库中,便于后续的数据分析和检索。数据库中的记录包括图片路径、识别到的关键点数据以及分类结果等信息。数据管理模块还提供查询功能,允许用户根据时间、标签、分类结果等条件进行检索,查询到的历史数据将通过AJAX响应返回前端,展示在用户界面上。

\end{enumerate}

这些模块协同工作,确保了系统能够高效地处理用户请求,上传图片,进行姿势识别,并将结果存储和展示。每个模块都可以独立扩展和优化,以适应不同的业务需求。


\subsubsection{接口设计}

系统使用Django REST framework (DRF) 提供RESTful API,确保前后端通信高效、灵活。以下是主要接口的设计及其详细说明:

\begin{enumerate}
    \item \textbf{POST /api/upload/}
    接收用户上传图片的请求,将图片保存到服务器,并返回图片存储路径及相关信息。
    \begin{itemize}
        \item \textbf{请求格式:}
        \begin{lstlisting}
        {
            "file": <binary_image_file>,
            "user_id": <int>
        }
        \end{lstlisting}
        \item \textbf{响应格式:}
        \begin{lstlisting}
        {
            "status": "success",
            "file_path": "/media/uploads/img_123.jpg",
            "message": "File uploaded successfully."
        }
        \end{lstlisting}
        \item \textbf{功能说明:}
        上传的图片文件会被存储在服务器的指定目录(如`/media/uploads/`)。系统会验证文件类型和大小,确保上传的文件为有效的图片格式(如JPEG、PNG)且大小在可接受范围内。如果验证失败,接口将返回错误信息。

    \end{itemize}

    \item \textbf{GET /api/results/}
    提供用户历史识别记录的查询接口,返回指定用户的上传图片及对应的姿势识别结果。
    \begin{itemize}
        \item \textbf{请求格式:}
        \begin{lstlisting}
        GET /api/results/?user_id=<int>&page=<int>&limit=<int>
        \end{lstlisting}
        \item \textbf{响应格式:}
        \begin{lstlisting}
        {
            "status": "success",
            "results": [
                {
                    "filename": "img_123.jpg",
                    "label": "Good",
                    "features": [...],
                    "timestamp": "2024-11-23T10:00:00Z"
                },
                ...
            ],
            "pagination": {
                "current_page": 1,
                "total_pages": 5
            }
        }
        \end{lstlisting}
        \item \textbf{功能说明:}
        系统支持分页返回历史记录,通过`page`和`limit`参数控制每页显示的条目数。接口可以根据用户身份过滤数据,确保每位用户仅能访问自己的历史记录。

    \end{itemize}

    \item \textbf{POST /api/recognize/}
    调用姿势识别程序,对上传的图片进行处理,并返回关键点数据及分类结果。
    \begin{itemize}
        \item \textbf{请求格式:}
        \begin{lstlisting}
        {
            "file_path": "/media/uploads/img_123.jpg",
            "user_id": <int>
        }
        \end{lstlisting}
        \item \textbf{响应格式:}
        \begin{lstlisting}
        {
            "status": "success",
            "filename": "img_123.jpg",
            "features": [
                {
                    "keypoint_id": 0,
                    "x": 0.566,
                    "y": 0.327,
                    "z": -1.393,
                    "visibility": 0.999
                },
                ...
            ],
            "label": "Good",
            "message": "Recognition completed successfully."
        }
        \end{lstlisting}
        \item \textbf{功能说明:}
        该接口调用姿势识别模块加载指定图片,提取33个关键点的位置信息(x, y, z)和可见性(visibility),并对图片姿势进行分类,返回分类结果(如“Good”、“Normal”、“Wrong”)。响应中包含完整的关键点信息和分类结果。

    \end{itemize}

\end{enumerate}

通过以上接口设计,系统实现了从图片上传到姿势识别、再到历史记录查询的完整闭环流程。API设计遵循RESTful原则,支持前端灵活调用,且易于扩展和维护。


\subsection{姿势识别程序整合}

\subsubsection{调用流程}

后端在接收到用户上传的图片后,将图片路径传递给姿势识别程序,程序完成姿势识别处理后返回分类结果和关键点数据。整个流程详细描述如下:

\begin{enumerate}
    \item \textbf{图片上传处理:}
    用户通过前端页面上传图片,后端接收图片后进行以下操作:
    \begin{itemize}
        \item 验证上传的文件格式(支持JPEG、PNG等)。
        \item 将图片保存至服务器指定目录(如`/media/uploads/`)。
        \item 返回图片路径供后续流程调用。
    \end{itemize}

    \item \textbf{调用姿势识别程序:}
    后端通过Python的\texttt{subprocess}模块调用姿势识别程序,执行以下步骤:
    \begin{itemize}
        \item 使用预定义的命令行接口,向姿势识别程序传递图片路径作为输入参数。例如:
        \begin{lstlisting}[language=Python]
        import subprocess
        result = subprocess.run(
            ["python", "pose_recognition.py", "--image", "/media/uploads/img_123.jpg"],
            capture_output=True,
            text=True
        )
        \end{lstlisting}
        \item 姿势识别程序加载预训练的模型,对输入图片进行关键点检测和分类。
        \item 程序生成一个包含识别结果的JSON文件,格式如下:
        \begin{lstlisting}
        {
            "filename": "img_123.jpg",
            "features": [
                {"keypoint_id": 0, "x": 0.566, "y": 0.327, "z": -1.393, "visibility": 0.999},
                ...
            ],
            "label": "Good"
        }
        \end{lstlisting}
        \item 姿势识别程序将处理状态(成功或失败)通过标准输出返回给后端。
    \end{itemize}

    \item \textbf{解析识别结果:}
    后端读取姿势识别程序生成的JSON文件,解析其中的关键点信息和分类结果。例如:
    \begin{lstlisting}[language=Python]
    import json
    with open('/output/results.json', 'r') as f:
        recognition_data = json.load(f)
    label = recognition_data['label']
    features = recognition_data['features']
    \end{lstlisting}

    \item \textbf{数据存储与返回:}
    解析后的数据会存入数据库,存储结构如下:
    \begin{itemize}
        \item 图片信息:包含文件名、上传时间和路径。
        \item 姿势识别结果:关键点数据(x, y, z, visibility)及分类标签。
    \end{itemize}
    后端通过API将结果返回给前端,供用户查看。返回示例如下:
    \begin{lstlisting}
    {
        "status": "success",
        "filename": "img_123.jpg",
        "label": "Good",
        "features": [
            {"keypoint_id": 0, "x": 0.566, "y": 0.327, "z": -1.393, "visibility": 0.999},
            ...
        ]
    }
    \end{lstlisting}

\end{enumerate}

上述调用流程实现了后端与姿势识别程序的无缝对接,确保数据流从上传到处理、再到存储与返回的完整性和高效性。


\subsubsection{数据格式}

姿势识别程序在完成图片处理后,将结果输出为JSON格式文件,以便后端解析和存储。以下为JSON文件的详细结构说明及示例:

\begin{itemize}
    \item \textbf{顶层字段描述:}
    \begin{itemize}
        \item \texttt{"image"}: 字符串类型,表示被处理图片的文件名。
        \item \texttt{"keypoints"}: 数组类型,包含关键点的检测结果,每个关键点由一组坐标和可见性组成。
        \item \texttt{"classification"}: 字符串类型,表示图片的分类结果。
    \end{itemize}

    \item \textbf{关键点字段描述:}
    每个关键点包含以下字段:
    \begin{itemize}
        \item \texttt{"id"}: 整数类型,关键点的唯一标识符。
        \item \texttt{"x"}: 浮点数类型,表示关键点在图片中的水平坐标(归一化至0-1范围)。
        \item \texttt{"y"}: 浮点数类型,表示关键点在图片中的垂直坐标(归一化至0-1范围)。
        \item \texttt{"z"}: 浮点数类型,表示关键点的深度值(相对摄像头的距离)。
        \item \texttt{"visibility"}: 浮点数类型,表示关键点的可见性置信度(范围为0-1)。
    \end{itemize}
\end{itemize}

\textbf{JSON文件示例}:

\begin{lstlisting}
{
    "image": "example.jpg",
    "keypoints": [
        {
            "id": 0,
            "x": 0.512,
            "y": 0.603,
            "z": -0.325,
            "visibility": 0.998
        },
        {
            "id": 1,
            "x": 0.435,
            "y": 0.725,
            "z": -0.210,
            "visibility": 0.965
        },
        ...
    ],
    "classification": "Good"
}
\end{lstlisting}

\textbf{字段说明}:
\begin{itemize}
    \item \texttt{"image"}: 文件名字段指向当前处理的图片,便于后端与数据库关联。
    \item \texttt{"keypoints"}: 数组中的每个对象对应一个人体关键点(如关节或部位),包含其三维坐标及置信度。
    \item \texttt{"classification"}: 结果分类字段,用于描述图片中的姿势类型,例如“Good”、“Normal”或“Wrong”。
\end{itemize}

该格式结构清晰,便于程序解析,同时为姿势识别结果的展示和存储提供了详细信息。


\subsection{数据库设计}

\subsubsection{数据表结构}

系统数据库设计包含以下主要数据表,每个表根据功能划分具体字段及约束条件,确保数据的完整性和可扩展性。

\begin{enumerate}
    \item \textbf{用户表(\texttt{users})} \\
    用于存储用户信息,支持用户登录和权限管理。
    \begin{table}[h!]
        \centering
        \begin{tabular}{|l|l|l|p{8cm}|}
            \hline
            \textbf{字段名} & \textbf{类型} & \textbf{约束} & \textbf{描述} \\
            \hline
            \texttt{id} & INTEGER & 主键,自增 & 用户唯一标识符 \\
            \texttt{username} & VARCHAR(50) & 唯一,非空 & 用户名,用于登录 \\
            \texttt{password} & VARCHAR(256) & 非空 & 加密存储的用户密码 \\
            \texttt{role} & VARCHAR(20) & 默认值为\texttt{'user'} & 用户权限级别,例如\texttt{'admin'}或\texttt{'user'} \\
            \hline
        \end{tabular}
        \caption{用户表结构}
        \label{tab:users}
    \end{table}

    \item \textbf{图片表(\texttt{images})} \\
    用于存储上传的图片及其相关信息。
    \begin{table}[h!]
        \centering
        \resizebox{\textwidth}{!}{
            \begin{tabular}{|l|l|l|p{8cm}|}
                \hline
                \textbf{字段名} & \textbf{类型} & \textbf{约束} & \textbf{描述} \\
                \hline
                \texttt{id} & INTEGER & 主键,自增 & 图片唯一标识符 \\
                \texttt{user\_id} & INTEGER & 外键,关联\texttt{users}表 & 上传图片的用户ID \\
                \texttt{file\_path} & TEXT & 非空 & 图片在服务器中的存储路径 \\
                \texttt{upload\_time} & TIMESTAMP & 默认当前时间 & 图片上传的时间 \\
                \hline
            \end{tabular}
        }
        \caption{图片表结构}
        \label{tab:images}
    \end{table}

    \item \textbf{识别结果表(\texttt{results})} \\
    用于存储姿势识别的结果数据,包括关键点信息和分类结果。
    \begin{table}[h!]
        \centering
        \resizebox{\textwidth}{!}{
        \begin{tabular}{|l|l|l|p{8cm}|}
            \hline
            \textbf{字段名} & \textbf{类型} & \textbf{约束} & \textbf{描述} \\
            \hline
            \texttt{id} & INTEGER & 主键,自增 & 识别结果唯一标识符 \\
            \texttt{image\_id} & INTEGER & 外键,关联\texttt{images}表 & 对应图片的ID \\
            \texttt{classification} & VARCHAR(20) & 非空 & 识别的分类结果,例如\texttt{'Good'} \\
            \texttt{keypoints} & JSONB & 非空 & 包含关键点的JSON数据,存储所有关键点的坐标和可见性 \\
            \texttt{processed\_time} & TIMESTAMP & 默认当前时间 & 识别完成的时间 \\
            \hline
        \end{tabular}
        }
        \caption{识别结果表结构}
        \label{tab:results}
    \end{table}
\end{enumerate}

\noindent
通过上述数据表设计,用户的基本信息、图片上传记录及姿势识别结果可以被高效存储和查询,同时为系统的扩展性提供基础支持。


\subsubsection{数据表示例}

\textbf{识别结果表}的字段设计如下所示:

\begin{table}[H]
    \centering
    \resizebox{\textwidth}{!}{
    \begin{tabular}{|c|c|p{8cm}|}
        \hline
        \textbf{字段名} & \textbf{类型} & \textbf{描述} \\
        \hline
        \texttt{id} & INTEGER & 主键,自增,用于唯一标识一条识别记录。 \\
        \texttt{image\_id} & ForeignKey & 外键,关联\texttt{images}表中的主键,用于标识对应的图片记录。 \\
        \texttt{keypoints} & JSON & 关键点数据,存储人体姿势的关键点信息,包括坐标和可见性。 \\
        \texttt{classification} & String & 分类结果,表示姿势识别的结果类别,例如\texttt{'Good'}、\texttt{'Blur'}等。 \\
        \texttt{created\_at} & DateTime & 记录创建时间,表示识别完成的时间戳。 \\
        \hline
    \end{tabular}
    }
    \caption{识别结果表字段设计}
    \label{table:result_table}
\end{table}

\noindent
以下是\textbf{识别结果表}中数据的一个示例:

\begin{lstlisting}[caption=识别结果表数据示例, label=listing:result_table_sample]
{
    "id": 123,
    "image_id": 45,
    "keypoints": [
        {"id": 0, "x": 0.5, "y": 0.6, "z": -0.3, "visibility": 0.99},
        {"id": 1, "x": 0.4, "y": 0.7, "z": -0.2, "visibility": 0.95},
        ...
    ],
    "classification": "Good",
    "created_at": "2024-11-23T15:45:00Z"
}
\end{lstlisting}

\noindent
通过该数据结构,系统能够快速查询、分析并呈现识别结果。


\subsection{部署方案}

\subsubsection{环境要求}

\begin{enumerate}
    \item \textbf{服务器:}建议使用Ubuntu 22.04操作系统。
    \item \textbf{运行环境:}安装Python 3.10及以上版本,配置Django及相关依赖。
    \item \textbf{数据库:}采用PostgreSQL 14。
\end{enumerate}

\subsubsection{部署步骤}

\begin{enumerate}
    \item 配置Django项目及应用。
    \item 设置Nginx和Gunicorn作为反向代理及应用服务器。
    \item 配置PostgreSQL数据库,并初始化数据表。
    \item 部署姿势识别程序及相关依赖。
\end{enumerate}
