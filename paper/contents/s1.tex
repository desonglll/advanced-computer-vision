% \section{数据集要求}

% \subsection{图片类别要求}

% 对于正常幼儿:拍摄婴儿自然的动作

% 对于异常幼儿:尽量捕捉其异常动作

% 每类尽可能多的采集不同幼儿的不同照片

% \textbf{总数要求}

% 正常类别:500-1000 张

% 异常类别:300-500 张(异常动作相对较难采集)

% 模糊类别:100-300 张(用作无效数据检测或异常检测模型)

% \textbf{单个数量要求}

% 新生儿:100-200 张

% 婴儿期(3-6个月):200-300 张

% 学步期(7-12个月):300-500 张

% 幼儿期(1-2岁):300-500 张

% 分为\textbf{正常、异样、模糊}三个文件夹采集照片
% \begin{figure}[H] % 控制图片的位置
%     \centering % 图片居中
%     \includegraphics[width=0.8\textwidth]{images/pic-tree-demo.jpg} % 设置宽度,插入图片
%     \caption{结构示例} % 添加图片标题
%     \label{fig:file-tree} % 给图片一个标签,方便引用
% \end{figure}
% \subsection{图片格式要求}

% \begin{enumerate}
%     \item 图片采用\textbf{16:9}的格式
%     \item 分辨率推荐\textbf{1080p (1920×1080)}以上
% \end{enumerate}
% % dataset/
% % ├── normal/    # 正常婴儿动作
% % ├── abnormal/  # 异常婴儿动作
% % └── blurry/    # 模糊图片(可能作为无效数据)



% \subsection{拍摄要求}

% \subsubsection{环境要求}

% \begin{enumerate}
%     \item 背景应尽量简单,避免复杂的背景干扰
%     \item 推荐使用\textbf{纯色背景}或室内环境,如床上、垫子上等
% \end{enumerate}

% \subsubsection{光照条件}

% \begin{enumerate}
%     \item 图像应在良好的光照条件下采集,避免过暗或过曝
%     \item 建议采集自然光和室内灯光条件下的图片,确保多样性
% \end{enumerate}

% \subsubsection{婴儿穿着要求}

% \begin{enumerate}
%     \item \textbf{避免}穿着影响婴儿身体骨架特征的\textbf{厚衣服}
%     \item 婴儿衣服颜色应与背景颜色形成\textbf{对比}
% \end{enumerate}

% \subsubsection{婴儿姿势要求}

% 数据集应该尽量包含以下姿势

% \begin{enumerate}
%     \item 睡姿(仰卧、俯卧、侧卧)
%     \item 爬行(手膝着地移动)
%     \item 坐姿(双手支撑、稳定坐姿)
%     \item 站姿(扶站、不扶站)
%     \item 躺卧翻滚(侧翻或转身)
% \end{enumerate}

% \subsubsection{拍摄视角要求}

% \begin{enumerate}
%     \item 多视角采集,覆盖从\textbf{上方、侧面、正面}等不同方向拍摄的图片
%     \item 尽量保持婴儿\textbf{全身在画面中},确保关键点(如肘关节、膝关节等)完整
% \end{enumerate}

% \subsection{图片示例}

% \begin{figure}[H] % 控制图片的位置
%     \centering % 图片居中
%     \includegraphics[width=0.8\textwidth]{images/example-baby.jpg} % 设置宽度,插入图片
%     \caption{正面示例} % 添加图片标题
%     \label{fig:front} % 给图片一个标签,方便引用
% \end{figure}
% \begin{figure}[H] % 控制图片的位置
%     \centering % 图片居中
%     \includegraphics[width=0.8\textwidth]{images/side.jpg} % 设置宽度,插入图片
%     \caption{侧面示例} % 添加图片标题
%     \label{fig:side} % 给图片一个标签,方便引用
% \end{figure}
% \begin{figure}[H] % 控制图片的位置
%     \centering % 图片居中
%     \includegraphics[width=0.8\textwidth]{images/lay.jpg} % 设置宽度,插入图片
%     \caption{顶面示例} % 添加图片标题
%     \label{fig:lay} % 给图片一个标签,方便引用
% \end{figure}

\newpage
\section{数据预处理与特征提取}

\subsection{图片文件处理}

为了生成带有骨架图的儿童图片,使用了OpenCV与MediaPipe Pose库的组合处理方法。以下是具体步骤:

\begin{enumerate}
    \item 图片读取与预处理:使用OpenCV读取输入的原始图片文件。通过函数 \texttt{cv2.imread()} 加载图片,并根据需要进行色彩空间转换(例如转换为RGB格式)。
    \item 骨架点检测与绘制:利用MediaPipe Pose模型对每张图片进行骨架点检测。通过模型的预测结果提取关键点坐标,并使用OpenCV的绘图函数(如 \texttt{cv2.line()} 和 \texttt{cv2.circle()})在原图上绘制骨架图。
    \item 文件命名规则:所有处理后的图片统一命名为 \texttt{img\_xxx.jpg} 格式,其中 \texttt{xxx} 为三位数字(例如 \texttt{img\_001.jpg}, \texttt{img\_002.jpg})。
    \item 分辨率与尺寸统一:对生成的图片进行尺寸归一化处理,确保所有图片的分辨率一致(例如,设定为 640×480)。使用 OpenCV 的 \texttt{cv2.resize()} 函数进行图片大小调整。
    \item 图片保存:将处理后的图片使用 \texttt{cv2.imwrite()} 保存到指定的目标文件夹中,确保文件格式为 JPEG。
\end{enumerate}

通过上述方法,可以高效生成标准化的儿童图片数据集,便于后续模型训练和评估。

\subsection{骨骼数据提取(单摄像头)}

为了提取人体的33个关键点信息,本研究采用MediaPipe Pose模块对单摄像头采集的图像进行处理,并将关键点数据保存为JSON格式文件。具体步骤如下:

\begin{enumerate}
    \item 人体关键点检测:使用MediaPipe Pose对输入图像进行人体骨骼关键点检测。该模型可以预测33个关键点的二维坐标(\texttt{x, y})及其置信度(\texttt{visibility}),这些点覆盖了人体主要的关节和骨骼部位。
    \item 数据结构设计:提取的关键信息被组织为JSON格式,结构设计如下:
    \begin{itemize}
        \item \texttt{image} 字段:存储图片的相对路径,格式为字符串(例如 \texttt{"images/img\_001.jpg"}),包含图片文件名以便追溯来源。
        \item \texttt{features} 字段:存储人体关键点的列表,每个关键点为一个JSON对象,包含以下子字段:
        \begin{itemize}
            \item \texttt{x}:关键点的水平坐标,值为归一化到[0, 1]范围的小数。
            \item \texttt{y}:关键点的垂直坐标,值为归一化到[0, 1]范围的小数。
            \item \texttt{visibility}:关键点的置信度,表示该点是否被可靠检测到,值范围为[0, 1]。
        \end{itemize}
    \end{itemize}
    \item JSON文件保存:每张图片对应一个JSON文件,文件名与图片文件名一致,仅扩展名为 \texttt{.json}(例如 \texttt{img\_001.json})。JSON文件被保存到指定的目录中。
    \item 代码实现细节:在提取过程中,首先使用MediaPipe的Pose检测API获取原始关键点信息,随后将坐标和置信度按上述结构存入字典。最后,使用Python的 \texttt{json} 模块将字典写入文件。
\end{enumerate}

以下为一个示例JSON文件的内容结构:
\begin{verbatim}
{
    "image": "images/img_001.jpg",
    "features": [
        {"x": 0.15, "y": 0.25, "visibility": 0.98},
        {"x": 0.20, "y": 0.30, "visibility": 0.95},
        ...
        {"x": 0.75, "y": 0.85, "visibility": 0.90}
    ]
}
\end{verbatim}

通过该方法,可以系统地将人体关键点数据提取并格式化保存,为后续动作分析或机器学习任务提供高质量的数据支持。

\subsection{数据格式(单摄像头)}

为了便于后续的数据处理和扩展功能,建议在完成数据标注操作后,进行人体骨骼关键点数据的提取和保存操作。提取的数据采用JSON格式进行组织,支持后续的追加和修改 \texttt{features} 属性。具体数据格式如下:

\begin{enumerate}
    \item 单个关键点的数据格式:每个关键点的数据包含以下属性:
    \begin{itemize}
        \item \texttt{keypoint\_id}:关键点的唯一编号,类型为整数,对应人体的特定位置(例如鼻子、肩膀等)。
        \item \texttt{x}:关键点的水平坐标,值为归一化到[0, 1]范围的小数。
        \item \texttt{y}:关键点的垂直坐标,值为归一化到[0, 1]范围的小数。
        \item \texttt{z}:关键点的深度坐标,值为相机坐标系中的相对深度信息,单位为标准化相机深度。
        \item \texttt{visibility}:关键点的置信度,表示检测结果的可靠性,值范围为[0, 1]。
    \end{itemize}

    单个关键点数据的JSON格式如下:
    \begin{lstlisting}
    Keypoint: {
        "keypoint_id": 0,
        "x": 0.5661032795906067,
        "y": 0.3272441029548645,
        "z": -1.3933179378509521,
        "visibility": 0.9999678134918213
    }
    \end{lstlisting}

    \item 单张图片的数据格式:每张图片对应一个JSON文件,文件中包含以下两个字段:
    \begin{itemize}
        \item \texttt{image}:图片的相对路径,格式为字符串(例如 \texttt{"images/train\_img.png"})。
        \item \texttt{features}:包含33个关键点的列表,每个关键点的数据格式与上述单个关键点一致。
    \end{itemize}

    单张图片数据的JSON格式如下:
    \begin{lstlisting}
    {
        "image": "images/train_img.png",
        "features": [
            {
                "keypoint_id": 0,
                "x": 0.5661032795906067,
                "y": 0.3272441029548645,
                "z": -1.3933179378509521,
                "visibility": 0.9999678134918213
            },
            {
                "keypoint_id": 1,
                "x": 0.4721031188964844,
                "y": 0.4152013659477234,
                "z": -1.4203872680664063,
                "visibility": 0.9998916387557983
            },
            ...
            {
                "keypoint_id": 32,
                "x": 0.518302321434021,
                "y": 0.65730202293396,
                "z": -0.8702016472816467,
                "visibility": 0.998772144317627
            }
        ]
    }
    \end{lstlisting}
\end{enumerate}

通过采用上述数据格式,可以清晰地描述每张图片中人体骨骼的关键点信息,同时保留关键信息的完整性,便于后续的数据分析和模型训练。

\section{数据标注}

\subsection{工具采用}

本项目采用 \textbf{doccano} 工具进行数据标注,选择 \textbf{Classification} 类型标注,用于对数据集中的图片或文本进行分类标注操作。doccano 是一个开源、用户友好的数据标注工具,支持多种标注类型,如文本分类、序列标注、实体关系标注等,非常适合本项目的数据标注需求。

为了提高数据标注的效率并支持多人协作,工具通过 Docker 部署,具体配置如下:
\begin{itemize}
    \item 使用官方提供的 doccano Docker 镜像。
    \item 配置服务运行时暴露 \textbf{8000端口},允许团队成员通过浏览器访问进行标注。
    \item 数据标注过程中,每位标注员可拥有独立账户,便于分配任务并管理标注进度。
    \item 支持标注结果的导出为 JSON 或 CSV 格式,以便后续的数据处理和模型训练使用。
\end{itemize}

Docker 部署的具体命令如下:
\begin{lstlisting}[language=bash]
# 拉取官方 doccano 镜像
docker pull doccano/doccano

# 运行 doccano 容器并暴露 8000 端口
docker run -d --name doccano -p 8000:8000 doccano/doccano

# 进入容器并创建管理员账户
docker exec -it doccano bash
# 在容器内运行以下命令
python manage.py createsuperuser
\end{lstlisting}

部署完成后,标注人员可通过 \texttt{http://localhost:8000} 或服务器对应的 IP 地址访问 doccano 界面,并登录完成数据标注任务。部署,暴露8000端口,方便多个人员进行数据标注。

\subsection{标签等级}

在数据标注过程中,标签(label)的设置具有灵活性,可根据具体任务需求选用以下两种分类方式之一:

\begin{enumerate}
    \item \textbf{数字分类:} 使用 \texttt{0-9} 共 \textbf{10个分类},适用于需要简单分类的场景,例如表示类别等级、评分或状态类型。
    \item \textbf{字母分类:} 使用 \texttt{a-z} 共 \textbf{26个分类},适用于分类类别较多或需要更多灵活性时,例如表示不同的特征类别或分组。
\end{enumerate}

\noindent 每张图片可以分配 \textbf{一个或多个标签}(multi-label classification),以准确表达图片的多种特征属性。例如:
\begin{itemize}
    \item 数字分类场景下,单张图片可以标注为 \texttt{[0, 3, 7]},表示其同时具有第 0、3、7 类特性。
    \item 字母分类场景下,单张图片可以标注为 \texttt{[a, c, g]},表示其对应的分类属性为 a、c、g。
\end{itemize}

\noindent 每个标签的分配应遵循以下原则:
\begin{itemize}
    \item 标签的数量和含义需要在标注任务开始前统一定义,并明确说明各标签的具体语义,以确保标注人员的理解一致性。
    \item 多标签标注适用于具有复杂属性的样本,允许对图片的多方面特性进行全面描述。
    \item 标注时,应使用 doccano 工具内的 \texttt{Multi-label Classification} 功能,支持在一张图片上同时分配多个标签。
\end{itemize}

\noindent 标注结果的格式如下:
\begin{lstlisting}
{
    "image": "train_img_001.png",
    "labels": ["a", "c", "g"]
}
{
    "image": "train_img_002.png",
    "labels": [0, 3, 7]
}
\end{lstlisting}

上述格式中:
\begin{itemize}
    \item \texttt{image} 字段表示图片路径,包含文件名。
    \item \texttt{labels} 字段表示图片的分类标签,可以是字母或数字。
\end{itemize}

此标签等级设计兼具简洁性和灵活性,适应不同复杂度的分类任务需求。

\subsection{数据格式}
在本项目中,数据格式为JSON格式,包含每个样本的基本信息。具体格式如下所示:

\begin{lstlisting}
{
    "id": 2,                    % 样本的唯一标识符
    "filename": "img_57.jpg",    % 图片文件名,包含路径
    "label": "Good",             % 图片的分类标签,表示其所归属的类别
    "Comments": []               % 图片的备注信息,可以为空数组或包含具体说明
}
\end{lstlisting}

\noindent 该格式中的各字段说明如下:

\begin{itemize}
    \item \texttt{id}:该字段为每个图片的唯一标识符,用于区分不同的样本。通常为一个整数,保证每个图片有一个唯一的编号。
    \item \texttt{filename}:该字段存储图片文件的名称,通常包括图片文件的相对路径或绝对路径,用于定位图片位置。以“.jpg”、“.png”等常见图像格式为后缀。
    \item \texttt{label}:该字段表示图片的分类标签或标注信息,描述图片属于哪个类别。在多分类场景下,可以为具体的类别名称(如“Good”),也可以使用数字、字母等分类编码。
    \item \texttt{Comments}:该字段用于存储与图片相关的备注信息,类型为数组。可以用于记录标注者的说明或图片的额外信息。如果没有备注信息,则为空数组(\texttt{[]})。
\end{itemize}

\noindent 该格式非常适合于分类任务的应用,可根据需要进一步扩展或修改其他字段,如加入标注时间、标注者信息等。通过此结构化的格式,能够方便地存储和管理大量图片数据,并支持后续的数据处理和分析。

\section{数据合并}

\subsection{合并工具选择}

在本项目中,使用Python的\texttt{json}模块对\texttt{json}和\texttt{jsonl}格式的数据进行合并处理。具体方法如下所述:

\begin{itemize}
    \item \texttt{json}格式文件:每个文件包含一个JSON对象,通常用于存储单个样本的完整信息。
    \item \texttt{jsonl}格式文件:每一行包含一个JSON对象,常用于存储多条样本信息,适合处理大量数据。
\end{itemize}

合并过程包括将多个\texttt{json}文件的内容合并为一个\texttt{jsonl}文件,或者将多个\texttt{jsonl}文件合并为一个大文件,方便后续的数据处理和分析。下面是一个合并操作的示例:

\begin{lstlisting}[language=Python]
import json

# 合并多个JSONL文件为一个JSON文件
def jsonl_to_json(jsonl_file_path, output_json_file_path):
    """
    将 JSONL 文件中的数据转换为标准的 JSON 格式文件。

    :param jsonl_file_path: JSONL 文件路径
    :param output_json_file_path: 输出的 JSON 文件路径
    """
    merged_data = []

    # 打开 JSONL 文件并逐行处理
    with open(jsonl_file_path, 'r') as jsonl_file:
        for line in jsonl_file:
            item = json.loads(line)  # 解析每一行的 JSON 对象
            merged_data.append(item)

    # 将合并后的数据写入输出 JSON 文件
    with open(output_json_file_path, 'w') as json_file:
        json.dump(merged_data, json_file, indent=4)

    print(f"Converted JSONL data saved to {output_json_file_path}")
\end{lstlisting}

\noindent 该代码使用了\texttt{json}模块的\texttt{json.load()}和\texttt{json.dump()}函数来读取和写入JSON数据。对于JSONL格式的文件,逐行读取每个JSON对象,并使用\texttt{json.loads()}进行解析。在合并过程中,\texttt{json.dump()}用于将合并后的数据输出为标准的JSON或JSONL格式。

通过这种方法,可以方便地处理大量数据,并且能够根据需要自由选择文件格式,确保数据处理的高效与灵活。

\subsection{预期合并格式}

在数据合并过程中,期望的格式为每个样本包含人体关键点信息与标注数据,最终合并后的数据结构示例如下所示:

\begin{lstlisting}
Keypoint: {
    "keypoint_id": 0,
    "x": 0.5661032795906067,
    "y": 0.3272441029548645,
    "z": -1.3933179378509521,
    "visibility": 0.9999678134918213
}
{
    "id": 2,
    "filename": "img_57.jpg",
    "features": [
        {
            "keypoint_id": 0,
            "x": 0.5661032795906067,
            "y": 0.3272441029548645,
            "z": -1.3933179378509521,
            "visibility": 0.9999678134918213
        },
        {
            "keypoint_id": 1,
            "x": 0.47492358136177063,
            "y": 0.29118308448791504,
            "z": -1.360788703918457,
            "visibility": 0.9999857540130615
        },
        ...
        {
            "keypoint_id": 32,
            "x": 0.2345515936613083,
            "y": 0.5407830471992493,
            "z": -1.2895463705062866,
            "visibility": 0.9999411101341248
        }
    ],
    "label": "Good",
    "Comments": []
}
\end{lstlisting}

\noindent 该数据结构包含以下字段:
\begin{itemize}
    \item \texttt{Keypoint}:每个关键点的位置信息,包括关键点的ID、三维坐标(\texttt{x}, \texttt{y}, \texttt{z})以及可见性(\texttt{visibility})。
    \item \texttt{id}:图片的唯一标识符。
    \item \texttt{filename}:图片文件名。
    \item \texttt{features}:包含33个关键点的列表,记录了人体各部位关键点的三维坐标与可见性信息。
    \item \texttt{label}:图片的标注标签(例如,"Good")。
    \item \texttt{Comments}:用户添加的评论或附加信息,默认为空列表。
\end{itemize}

\noindent 此格式通过\texttt{features}字段存储每张图片的关键点数据,并通过\texttt{label}字段进行分类标注,能够帮助我们在后续分析和模型训练中使用。这种结构便于后续处理与进一步的特征提取。


\subsection{代码片段}

\section{模型训练}

\subsection{工具选择}

在本项目中,选择TensorFlow作为主要的深度学习框架进行模型训练。TensorFlow是一个开源的机器学习框架,广泛应用于深度学习、神经网络训练和推理,具有高度的灵活性和扩展性。其优点包括:

\begin{itemize}
    \item \textbf{广泛的社区支持}:TensorFlow拥有庞大的开发者社区和丰富的文档资源,使得开发者能够快速解决遇到的问题。
    \item \textbf{高效的性能优化}:TensorFlow支持GPU和TPU加速训练,可以大幅提高训练速度,特别是在处理大量数据时。
    \item \textbf{跨平台支持}:TensorFlow支持多种平台,包括Windows、Linux、macOS和移动设备,能够在各种硬件上运行。
    \item \textbf{模型部署便捷}:TensorFlow的模型可以方便地部署到多种设备,包括服务器、移动设备以及嵌入式系统,支持TensorFlow Lite和TensorFlow.js等工具进行边缘计算和浏览器端推理。
    \item \textbf{丰富的API支持}:TensorFlow提供了丰富的API接口,支持Python、C++、Java等多种编程语言,用户可以根据实际需求选择合适的API进行开发。
\end{itemize}

在使用TensorFlow进行模型训练时,主要采用TensorFlow Keras API进行神经网络的构建与训练,Keras提供了一个简洁的接口,能够帮助开发者快速构建、训练和评估深度学习模型。

通过TensorFlow的自动求导机制和优化器支持,可以实现高效的反向传播和梯度下降,优化模型的参数,减少训练时间。此外,TensorFlow支持集成式训练流程,包括数据预处理、模型评估和部署,使得整个机器学习管道更加流畅。

选择TensorFlow不仅能够满足高效训练的需求,还能利用其强大的模型扩展性,方便后期模型的优化与部署。


\subsection{传入数据}

\begin{lstlisting}
def load_data(json_path):
with open(json_path, 'r') as f:
    data = json.load(f)

inputs = []
labels = []
for sample in data:
    features = sample["features"]
    # Flatten each (x, y, z, visibility) into a single array
    flattened_features = []
    for keypoint in features:
        flattened_features.extend([keypoint["x"], keypoint["y"], keypoint["z"], keypoint["visibility"]])

    inputs.append(flattened_features)

    # Multi-label processing: Convert label list to a binary vector
    label_list = sample["label"]
    labels.append(label_list)

return np.array(inputs), labels
\end{lstlisting}

\subsection{数据规范}

\begin{lstlisting}
# Load data
X, y = load_data('../data.json')
print(f"Initial X shape: {X.shape}")
ALL_LABELS = []
for label in y:
    if label not in ALL_LABELS:
        ALL_LABELS.append(label)
print(f"ALL LABELS: {ALL_LABELS}")
# N categories
N = len(ALL_LABELS)
print(f"N: {N}")
KEY_POINTS = 33
FEATURES = 4
# Reshape data to fit Conv1D input: (samples, steps, features)
X = X.reshape((X.shape[0], KEY_POINTS, 4))  # 33 keypoints with 4 features (x, y, z, visibility)
# Checking the shape of the reshaped data
print(X.shape)
\end{lstlisting}

\subsection{OneHotEncoding编码}

\begin{lstlisting}
from sklearn.preprocessing import LabelEncoder
from tensorflow.keras.utils import to_categorical
# Initialize a LabelEncoder to convert strings to integers
label_encoder = LabelEncoder()
# Fit and transform the labels to integers
y_int = label_encoder.fit_transform(y)
# Now apply to_categorical for one-hot encoding
y_onehot = to_categorical(y_int, num_classes=len(label_encoder.classes_))
print(f"One-hot encoded labels shape: {y_onehot.shape}")
print(y_onehot)
\end{lstlisting}

\subsection{模型定义}

\begin{lstlisting}
import tensorflow as tf
from tensorflow.keras.models import Sequential
from tensorflow.keras.layers import Conv1D, MaxPooling1D, Flatten, Dense, Dropout
# Define the model
model = Sequential()
# Add Conv1D layer
model.add(Conv1D(filters=64, kernel_size=3, activation='relu', input_shape=(KEY_POINTS, FEATURES)))
# Add MaxPooling1D layer
model.add(MaxPooling1D(pool_size=2))
# Add another Conv1D layer
model.add(Conv1D(filters=128, kernel_size=3, activation='relu'))
# Add another MaxPooling1D layer
model.add(MaxPooling1D(pool_size=2))
# Flatten the output from Conv1D layers
model.add(Flatten())
# Add Dense layer with dropout for regularization
model.add(Dense(128, activation='relu'))
model.add(Dropout(0.5))
# Output layer with softmax activation (for classification)
model.add(Dense(N, activation='softmax'))  # N is the number of classes
\end{lstlisting}

\subsection{模型编译}

\begin{lstlisting}
# Compile the model
model.compile(optimizer='adam', loss='categorical_crossentropy', metrics=['accuracy'])
# Summary of the model
model.summary()
\end{lstlisting}

\subsection{训练模型}

\begin{lstlisting}
# Train the Model
history = model.fit(X, y_onehot, epochs=10, batch_size=32, validation_split=0.2)
# Evaluate the Model
loss, accuracy = model.evaluate(X, y_onehot)
print(f'Accuracy: {accuracy * 100:.2f}%')
\end{lstlisting}

\section{预测图片}

\subsection{提取单张图片的关键点}

\begin{lstlisting}
import cv2
from mediapipe_impl.pose_estimation import PoseEstimationModule as pm
detector = pm.PoseDetector()
def extract_keypoints(image_path):
    cap = cv2.VideoCapture(image_path)
    success, img = cap.read()
    img = detector.find_pose(img=img)
    lm_list = detector.find_position(img, draw=False)
    return lm_list
\end{lstlisting}

\subsection{预测图片}

\begin{lstlisting}
def predict_image(image_path):
    data = []
    keypoints = extract_keypoints(image_path)
    for keypoint in keypoints:
        data.extend([keypoint["x"], keypoint["y"], keypoint["z"], keypoint["visibility"]])
    X = np.array(data).reshape((1, 33, 4))
    print(X.shape)
    # 模型预测
    predictions = model.predict(X)

    # 获取预测类别的索引
    predicted_class = np.argmax(predictions, axis=1)[0]

    print(f"Predicted Class: {predicted_class}")
    print(ALL_LABELS[predicted_class])


# 示例:预测一张新图像
image_path = '../datasets/img.jpg'
result = predict_image(image_path)
\end{lstlisting}

\section{定期保存角度信息}

程序运行期间定期保存给定三个 keypoint\_id 的角度信息

\begin{lstlisting}
def process_and_save_angle(self, img, p1, p2, p3, save_interval, filepath="angles.csv", draw=True):
    """
    处理图像,计算角度,并定期保存角度和时间戳到CSV文件中。
    :param img: 输入图像
    :param p1: 第一个关键点ID
    :param p2: 第二个关键点ID
    :param p3: 第三个关键点ID
    :param save_interval: 保存间隔时间(秒)
    :param filepath: CSV文件路径
    :param draw: 是否在图像上绘制关键点和角度
    :return: 处理后的图像
    """
    # 初始化静态变量,用于存储上次保存时间
    if not hasattr(self, "_last_save_time"):
        self._last_save_time = 0

    lm_list = self.find_position(img, draw=False)

    if len(lm_list) > 0:
        # 计算三个关键点的角度
        angle = self.find_angle(img, p1, p2, p3, draw=draw)

        # 获取当前时间
        current_time = time.time()

        # 检查是否需要保存角度
        if current_time - self._last_save_time >= save_interval:
            # 保存到CSV文件
            with open(filepath, mode='a', newline='') as file:
                writer = csv.writer(file)
                writer.writerow([time.strftime('%Y-%m-%d %H:%M:%S', time.localtime(current_time)), angle])
            print(
                f"Angle {angle:.2f} saved at time {time.strftime('%Y-%m-%d %H:%M:%S', time.localtime(current_time))}.")

            # 更新保存时间
            self._last_save_time = current_time

    return img
\end{lstlisting}

\section{实际运行}

以下是使用Python进行姿势估计和模型预测的完整代码实现,采用了OpenCV、TensorFlow和MediaPipe框架进行实时视频处理与姿势分类。

\begin{lstlisting}[language=Python]
import numpy as np
import time
from tensorflow.keras.models import load_model
import cv2
from mediapipe_impl.pose_estimation import PoseEstimationModule as pm
import mediapipe as mp

# 初始化MediaPipe和姿势估计模块
mp_draw = mp.solutions.drawing_utils
mp_pose = mp.solutions.pose
pose = mp_pose.Pose()
detector = pm.PoseDetector()

# 加载已保存的模型
model = load_model('./models/pose_estimation.keras')

# 验证模型是否加载成功
model.summary()

# 定义标签
ALL_LABELS = ['Blur', 'Normal', 'Wrong']

# 打开视频摄像头
cap = cv2.VideoCapture(1)

# 初始化时间变量以计算帧率
p_time = 0
while True:
    # 读取每一帧图像
    success, img = cap.read()
    img_rgb = cv2.cvtColor(img, cv2.COLOR_BGR2RGB)
    results = pose.process(img_rgb)

    # 检测到姿势标记时进行处理
    if results.pose_landmarks:
        # 绘制姿势的骨架图
        mp_draw.draw_landmarks(img, results.pose_landmarks, mp_pose.POSE_CONNECTIONS)
        data = []
        # 提取每个关键点的位置和可见性
        for id, lm in enumerate(results.pose_landmarks.landmark):
            h, w, c = img.shape
            data.extend([lm.x, lm.y, lm.z, lm.visibility])
            cx, cy = int(lm.x * w), int(lm.y * h)
            # 在图像上绘制每个关键点
            cv2.circle(img, (cx, cy), 10, (255, 0, 0), cv2.FILLED)
        # 将关键点数据转化为适合模型输入的格式
        X = np.array(data).reshape((1, 33, 4))
        print(X.shape)

        # 对提取的特征数据进行预测
        predictions = model.predict(X)

        # 获取预测类别的索引
        predicted_class = np.argmax(predictions, axis=1)[0]
        print(f"Predicted Class: {predicted_class}")
        print(ALL_LABELS[predicted_class])

    # 计算每秒帧数 (FPS)
    c_time = time.time()
    fps = 1 / (c_time - p_time)
    p_time = c_time

    # 在图像上显示帧率
    cv2.putText(img, 'FPS: {:.2f}'.format(int(fps)), (70, 50), cv2.FONT_HERSHEY_PLAIN, 3, (255, 0, 0), 3)

    # 显示处理后的图像
    cv2.imshow('img', img)
    cv2.waitKey(1)
\end{lstlisting}

\subsection{代码说明}

1. 导入必要的库:首先,代码导入了$numpy$用于数据处理,$time$用于计算帧率,$tensorflow.keras.models$用于加载已训练的模型,$cv2$用于图像处理,$mediapipe$用于姿势估计。

2. 模型加载与验证:使用$load\_model$方法加载预训练的姿势估计模型,并通过$model.summary()$查看模型架构。

3. 视频捕获与图像处理:
   - 使用$cv2.VideoCapture(1)$启动摄像头并捕获每一帧。
   - 将每一帧从BGR格式转换为RGB格式,以便传递给MediaPipe进行处理。
   - 使用$pose.process$处理每帧图像,提取人体姿势标记(关键点)。

4. 姿势估计与关键点提取:
   - 如果图像中检测到姿势标记,通过$mp\_draw.draw\_landmarks$绘制姿势的骨架图。
   - 从$results.pose\_landmarks.landmark$中提取每个关键点的坐标(x, y, z)及可见性(visibility)。
   - 将所有关键点的坐标和可见性信息存储在$data$列表中,并转换为适合模型输入的格式(形状为$(1, 33, 4)$)。

5. 模型预测:
   - 使用$model.predict$进行预测,获取类别的概率分布。
   - 通过$np.argmax$找到概率最大的类别索引,并输出预测结果及相应的标签。

6. 帧率计算与显示:
   - 使用$time.time()$计算每帧处理所需的时间,并计算每秒帧数(FPS)。
   - 使用$cv2.putText$在图像上显示当前FPS。

7. 显示图像:使用$cv2.imshow$实时显示带有骨架图和FPS信息的处理后图像。

通过此实现,系统能够实时捕捉人体姿势并进行分类,适用于视频监控、健身训练等场景的姿势分析和反馈。
