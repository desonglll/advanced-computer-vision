% \section{数据集要求}

% \subsection{图片类别要求}

% 对于正常幼儿:拍摄婴儿自然的动作

% 对于异常幼儿:尽量捕捉其异常动作

% 每类尽可能多的采集不同幼儿的不同照片

% \textbf{总数要求}

% 正常类别:500-1000 张

% 异常类别:300-500 张(异常动作相对较难采集)

% 模糊类别:100-300 张(用作无效数据检测或异常检测模型)

% \textbf{单个数量要求}

% 新生儿:100-200 张

% 婴儿期(3-6个月):200-300 张

% 学步期(7-12个月):300-500 张

% 幼儿期(1-2岁):300-500 张

% 分为\textbf{正常、异样、模糊}三个文件夹采集照片
% \begin{figure}[H] % 控制图片的位置
%     \centering % 图片居中
%     \includegraphics[width=0.8\textwidth]{images/pic-tree-demo.jpg} % 设置宽度,插入图片
%     \caption{结构示例} % 添加图片标题
%     \label{fig:file-tree} % 给图片一个标签,方便引用
% \end{figure}
% \subsection{图片格式要求}

% \begin{enumerate}
%     \item 图片采用\textbf{16:9}的格式
%     \item 分辨率推荐\textbf{1080p (1920×1080)}以上
% \end{enumerate}
% % dataset/
% % ├── normal/    # 正常婴儿动作
% % ├── abnormal/  # 异常婴儿动作
% % └── blurry/    # 模糊图片(可能作为无效数据)



% \subsection{拍摄要求}

% \subsubsection{环境要求}

% \begin{enumerate}
%     \item 背景应尽量简单,避免复杂的背景干扰
%     \item 推荐使用\textbf{纯色背景}或室内环境,如床上、垫子上等
% \end{enumerate}

% \subsubsection{光照条件}

% \begin{enumerate}
%     \item 图像应在良好的光照条件下采集,避免过暗或过曝
%     \item 建议采集自然光和室内灯光条件下的图片,确保多样性
% \end{enumerate}

% \subsubsection{婴儿穿着要求}

% \begin{enumerate}
%     \item \textbf{避免}穿着影响婴儿身体骨架特征的\textbf{厚衣服}
%     \item 婴儿衣服颜色应与背景颜色形成\textbf{对比}
% \end{enumerate}

% \subsubsection{婴儿姿势要求}

% 数据集应该尽量包含以下姿势

% \begin{enumerate}
%     \item 睡姿(仰卧、俯卧、侧卧)
%     \item 爬行(手膝着地移动)
%     \item 坐姿(双手支撑、稳定坐姿)
%     \item 站姿(扶站、不扶站)
%     \item 躺卧翻滚(侧翻或转身)
% \end{enumerate}

% \subsubsection{拍摄视角要求}

% \begin{enumerate}
%     \item 多视角采集,覆盖从\textbf{上方、侧面、正面}等不同方向拍摄的图片
%     \item 尽量保持婴儿\textbf{全身在画面中},确保关键点(如肘关节、膝关节等)完整
% \end{enumerate}

% \subsection{图片示例}

% \begin{figure}[H] % 控制图片的位置
%     \centering % 图片居中
%     \includegraphics[width=0.8\textwidth]{images/example-baby.jpg} % 设置宽度,插入图片
%     \caption{正面示例} % 添加图片标题
%     \label{fig:front} % 给图片一个标签,方便引用
% \end{figure}
% \begin{figure}[H] % 控制图片的位置
%     \centering % 图片居中
%     \includegraphics[width=0.8\textwidth]{images/side.jpg} % 设置宽度,插入图片
%     \caption{侧面示例} % 添加图片标题
%     \label{fig:side} % 给图片一个标签,方便引用
% \end{figure}
% \begin{figure}[H] % 控制图片的位置
%     \centering % 图片居中
%     \includegraphics[width=0.8\textwidth]{images/lay.jpg} % 设置宽度,插入图片
%     \caption{顶面示例} % 添加图片标题
%     \label{fig:lay} % 给图片一个标签,方便引用
% \end{figure}

\newpage
\section{数据预处理与特征提取}

\subsection{图片文件处理}

使用OpenCV配合MediaPipePose,生成带有骨架图的儿童图片,统一命名img\_xxx.jpg,统一分辨率以及图片大小。

\subsection{骨骼数据提取(单摄像头)}

使用MediaPipe进行人体33个关键点的提取,保存为json格式,内有image和features字段。
image字段代表图片的路径,包含图片的名字。
features字段包含33个关键点的keypoint信息。

\subsection{数据格式(单摄像头)}

建议提取数据放在数据标注操作之后,方便json数据的追加features属性。
1. 单个关键点的数据格式:
\begin{lstlisting}
Keypoint: {
    "keypoint_id": 0,
    "x": 0.5661032795906067,
    "y": 0.3272441029548645,
    "z": -1.3933179378509521,
    "visibility": 0.9999678134918213
}
\end{lstlisting}
2. 单个图片的数据格式(带有33个关键点):
\begin{lstlisting}
{
    "image": "train_img.png",
    "features": [Keypoint]
}
\end{lstlisting}

\section{数据标注}

\subsection{工具采用}

使用doccano工具进行Classification类型标注。
使用docker部署,暴露8000端口,方便多个人员进行数据标注。

\subsection{标签等级}

可以选用0-9十个分类或者a-z的字母分类作为label。
每个图片可以有多个label。

\subsection{数据格式}

\begin{lstlisting}
{
    "id": 2,
    "filename": "img_57.jpg",
    "label": "Good",
    "Comments": []
}
\end{lstlisting}

\section{数据合并}

\subsection{合并工具选择}

使用Python的json模块进行json与jsonl数据的合并处理。

\subsection{预期合并格式}
\begin{lstlisting}
Keypoint: {
    "keypoint_id": 0,
    "x": 0.5661032795906067,
    "y": 0.3272441029548645,
    "z": -1.3933179378509521,
    "visibility": 0.9999678134918213
}
{
    "id": 2,
    "filename": "img_57.jpg",
    "features": [Keypoint]
    "label": "Good",
    "Comments": []
}
\end{lstlisting}

\subsection{代码片段}

\section{模型训练}

\subsection{工具选择}

选用Tensorflow进行训练。

\subsection{传入数据}

\begin{lstlisting}
def load_data(json_path):
with open(json_path, 'r') as f:
    data = json.load(f)

inputs = []
labels = []
for sample in data:
    features = sample["features"]
    # Flatten each (x, y, z, visibility) into a single array
    flattened_features = []
    for keypoint in features:
        flattened_features.extend([keypoint["x"], keypoint["y"], keypoint["z"], keypoint["visibility"]])

    inputs.append(flattened_features)

    # Multi-label processing: Convert label list to a binary vector
    label_list = sample["label"]
    labels.append(label_list)

return np.array(inputs), labels
\end{lstlisting}

\subsection{数据规范}

\begin{lstlisting}
# Load data
X, y = load_data('../data.json')
print(f"Initial X shape: {X.shape}")
ALL_LABELS = []
for label in y:
    if label not in ALL_LABELS:
        ALL_LABELS.append(label)
print(f"ALL LABELS: {ALL_LABELS}")
# N categories
N = len(ALL_LABELS)
print(f"N: {N}")
KEY_POINTS = 33
FEATURES = 4
# Reshape data to fit Conv1D input: (samples, steps, features)
X = X.reshape((X.shape[0], KEY_POINTS, 4))  # 33 keypoints with 4 features (x, y, z, visibility)
# Checking the shape of the reshaped data
print(X.shape)
\end{lstlisting}

\subsection{OneHotEncoding编码}

\begin{lstlisting}
from sklearn.preprocessing import LabelEncoder
from tensorflow.keras.utils import to_categorical
# Initialize a LabelEncoder to convert strings to integers
label_encoder = LabelEncoder()
# Fit and transform the labels to integers
y_int = label_encoder.fit_transform(y)
# Now apply to_categorical for one-hot encoding
y_onehot = to_categorical(y_int, num_classes=len(label_encoder.classes_))
print(f"One-hot encoded labels shape: {y_onehot.shape}")
print(y_onehot)
\end{lstlisting}

\subsection{模型定义}

\begin{lstlisting}
import tensorflow as tf
from tensorflow.keras.models import Sequential
from tensorflow.keras.layers import Conv1D, MaxPooling1D, Flatten, Dense, Dropout
# Define the model
model = Sequential()
# Add Conv1D layer
model.add(Conv1D(filters=64, kernel_size=3, activation='relu', input_shape=(KEY_POINTS, FEATURES)))
# Add MaxPooling1D layer
model.add(MaxPooling1D(pool_size=2))
# Add another Conv1D layer
model.add(Conv1D(filters=128, kernel_size=3, activation='relu'))
# Add another MaxPooling1D layer
model.add(MaxPooling1D(pool_size=2))
# Flatten the output from Conv1D layers
model.add(Flatten())
# Add Dense layer with dropout for regularization
model.add(Dense(128, activation='relu'))
model.add(Dropout(0.5))
# Output layer with softmax activation (for classification)
model.add(Dense(N, activation='softmax'))  # N is the number of classes
\end{lstlisting}

\subsection{模型编译}

\begin{lstlisting}
# Compile the model
model.compile(optimizer='adam', loss='categorical_crossentropy', metrics=['accuracy'])
# Summary of the model
model.summary()
\end{lstlisting}

\subsection{训练模型}

\begin{lstlisting}
# Train the Model
history = model.fit(X, y_onehot, epochs=10, batch_size=32, validation_split=0.2)
# Evaluate the Model
loss, accuracy = model.evaluate(X, y_onehot)
print(f'Accuracy: {accuracy * 100:.2f}%')
\end{lstlisting}

\section{预测图片}

\subsection{提取单张图片的关键点}

\begin{lstlisting}
import cv2
from mediapipe_impl.pose_estimation import PoseEstimationModule as pm
detector = pm.PoseDetector()
def extract_keypoints(image_path):
    cap = cv2.VideoCapture(image_path)
    success, img = cap.read()
    img = detector.find_pose(img=img)
    lm_list = detector.find_position(img, draw=False)
    return lm_list
\end{lstlisting}

\subsection{预测图片}

\begin{lstlisting}
def predict_image(image_path):
    data = []
    keypoints = extract_keypoints(image_path)
    for keypoint in keypoints:
        data.extend([keypoint["x"], keypoint["y"], keypoint["z"], keypoint["visibility"]])
    X = np.array(data).reshape((1, 33, 4))
    print(X.shape)
    # 模型预测
    predictions = model.predict(X)

    # 获取预测类别的索引
    predicted_class = np.argmax(predictions, axis=1)[0]

    print(f"Predicted Class: {predicted_class}")
    print(ALL_LABELS[predicted_class])


# 示例:预测一张新图像
image_path = '../datasets/img.jpg'
result = predict_image(image_path)
\end{lstlisting}

\section{定期保存角度信息}

程序运行期间定期保存给定三个 keypoint\_id 的角度信息

\begin{lstlisting}
def process_and_save_angle(self, img, p1, p2, p3, save_interval, filepath="angles.csv", draw=True):
    """
    处理图像,计算角度,并定期保存角度和时间戳到CSV文件中。
    :param img: 输入图像
    :param p1: 第一个关键点ID
    :param p2: 第二个关键点ID
    :param p3: 第三个关键点ID
    :param save_interval: 保存间隔时间(秒)
    :param filepath: CSV文件路径
    :param draw: 是否在图像上绘制关键点和角度
    :return: 处理后的图像
    """
    # 初始化静态变量,用于存储上次保存时间
    if not hasattr(self, "_last_save_time"):
        self._last_save_time = 0

    lm_list = self.find_position(img, draw=False)

    if len(lm_list) > 0:
        # 计算三个关键点的角度
        angle = self.find_angle(img, p1, p2, p3, draw=draw)

        # 获取当前时间
        current_time = time.time()

        # 检查是否需要保存角度
        if current_time - self._last_save_time >= save_interval:
            # 保存到CSV文件
            with open(filepath, mode='a', newline='') as file:
                writer = csv.writer(file)
                writer.writerow([time.strftime('%Y-%m-%d %H:%M:%S', time.localtime(current_time)), angle])
            print(
                f"Angle {angle:.2f} saved at time {time.strftime('%Y-%m-%d %H:%M:%S', time.localtime(current_time))}.")

            # 更新保存时间
            self._last_save_time = current_time

    return img
\end{lstlisting}