\section{数据集要求}

\subsection{图片类别要求}

对于正常幼儿:拍摄婴儿自然的动作。

对于异常幼儿:尽量捕捉其异常动作。

每类尽可能多的采集不同幼儿的不同照片。

\textbf{总数要求}

正常类别:500-1000 张

异常类别:300-500 张(异常动作相对较难采集)

模糊类别:100-300 张(用作无效数据检测或异常检测模型)

\textbf{单个数量要求}

新生儿:100-200 张

婴儿期(3-6个月):200-300 张

学步期(7-12个月):300-500 张

幼儿期(1-2岁):300-500 张

分为\textbf{正常、异样、模糊}三个文件夹采集照片。

\begin{figure}[H]
    \centering
    \includegraphics[width=0.8\textwidth]{images/pic-tree-demo.jpg}
    \caption{结构示例}
    \label{fig:file-tree}
\end{figure}

\subsection{图片格式要求}

\begin{enumerate}
    \item 图片采用\textbf{16:9}的格式。
    \item 分辨率推荐\textbf{1080p (1920×1080)}以上。
\end{enumerate}

% \texttt{dataset/}
% \begin{itemize}
%     \item \texttt{normal/}     正常婴儿动作
%     \item \texttt{abnormal/}   异常婴儿动作
%     \item \texttt{blurry/}     模糊图片(可能作为无效数据)
% \end{itemize}

\subsection{拍摄要求}

\subsubsection{环境要求}

\begin{enumerate}
    \item 背景应尽量简单,避免复杂的背景干扰。
    \item 推荐使用\textbf{纯色背景}或室内环境,如床上、垫子上等。
\end{enumerate}

\subsubsection{光照条件}

\begin{enumerate}
    \item 图像应在良好的光照条件下采集,避免过暗或过曝。
    \item 建议采集自然光和室内灯光条件下的图片,确保多样性。
\end{enumerate}

\subsubsection{婴儿穿着要求}

\begin{enumerate}
    \item \textbf{避免}穿着影响婴儿身体骨架特征的\textbf{厚衣服}。
    \item 婴儿衣服颜色应与背景颜色形成\textbf{对比}。
\end{enumerate}

\subsubsection{婴儿姿势要求}

数据集应该尽量包含以下姿势:

\begin{enumerate}
    \item 睡姿(仰卧、俯卧、侧卧)
    \item 爬行(手膝着地移动)
    \item 坐姿(双手支撑、稳定坐姿)
    \item 站姿(扶站、不扶站)
    \item 躺卧翻滚(侧翻或转身)
\end{enumerate}

\subsubsection{拍摄视角要求}

\begin{enumerate}
    \item 多视角采集,覆盖从\textbf{上方、侧面、正面}等不同方向拍摄的图片。
    \item 尽量保持婴儿\textbf{全身在画面中},确保关键点(如肘关节、膝关节等)完整。
\end{enumerate}

\subsection{图片示例}

\begin{figure}[H]
    \centering
    \includegraphics[width=0.8\textwidth]{images/example-baby.jpg}
    \caption{正面示例}
    \label{fig:front}
\end{figure}

\begin{figure}[H]
    \centering
    \includegraphics[width=0.8\textwidth]{images/side.jpg}
    \caption{侧面示例}
    \label{fig:side}
\end{figure}

\begin{figure}[H]
    \centering
    \includegraphics[width=0.8\textwidth]{images/lay.jpg} 
    \caption{顶面示例}
    \label{fig:lay}
\end{figure}
